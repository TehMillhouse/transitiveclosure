\documentclass[12pt,a4paper,twoside]{article}

\usepackage[utf8]{inputenc}
\usepackage[T1]{fontenc}

\usepackage{amsmath}
\usepackage{algorithm, algpseudocode}

\DeclareMathOperator*{\argmin}{\arg\!\min}

\begin{document}

\title{Fast Transitive Closure Computation}
\author{Max Wagner \and Sebastian Ullrich \and Tobias Fleck}
\date{August 2014}
\maketitle

\def\sectionautorefname{Section}
\def\subsectionautorefname{Subsection}

\newcommand\ie{i.e.\ }
\newcommand\eg{e.g.\ }
\newcommand\credits[1]{\begin{flushright}\emph{#1}\end{flushright}}

\begin{abstract}
  In this paper, we discuss and evaluate various algorithms for computing the transitive closure of a directed acyclic graph. The algorithms are implemented on top of a generic, template-based framework to make them independent from the input and output representation, where possible.
\end{abstract}

\section{Introduction}

\section{The Graph Framework}

\section{Basic Algorithms}
\credits{Floyd-Warshall: Tobias Fleck \\ BFS, DFS, TLS: Sebastian Ullrich}

Both breadth-first search (BFS) and depth-first search (DFS) are well-known algorithms for computing the graph reachability starting from a single source node. Topological level search (TLS) can be regarded as a variant of BFS where nodes are visited in order of increasing topological level. A pseudo code implementation is shown in Algorithm~\ref{algo:TLS}.

\begin{algorithm}
  \begin{algorithmic}
    \Function{TLS}{s}
      \State find topological levels
      \State $next[level(s)] \gets \{s\}$
      \For{$l \gets level(s), maxLevel$}
        \For{$u \in next[l]$}
          \State $visited[u] \gets false$
          \For{$v \in V, (u,v) \in E$}
            \If{$\neg visited[v]$}
              \State $visited[v] \gets true$
              \State $reached[s, v] \gets true$
              \State $next[level(v)] \gets next[level(v)] \cup \{v\}$
            \EndIf
          \EndFor
        \EndFor
        \State $next[l] \gets \emptyset$ \Comment Optimization: reset for next invocation
      \EndFor
    \EndFunction
  \end{algorithmic}
  \caption{Topological Level Search}
  \label{algo:TLS}
\end{algorithm}

By separately computing reachability for each node in the graph, these single-source algorithms can compute the transitive closure in $\mathcal{O}(n(n+m))$. As a small optimization, we can avoid having to reset the whole $visited$ array by saving the reached nodes in a list in addition to the $reached$ bit array and explicitly resetting only those nodes. This is even easier for TLS, as evidenced by Algorithm~\ref{algo:TLS}, because a node cannot be reached any more once it has been visited.

\section{Bit-Parallel TLS}

As a first optimization over the single-source algorithms, we can reduce the expected number of graph traversals by generalizing the algorithms to traverse a whole set $S$ of start nodes at the same time; Algorithm~\ref{algo:MultiTLS} shows such a straightforward extension for TLS. Selecting the nodes in $S$ from the same topological level is a good heuristics for finding nodes whose reachable subgraphs overlap maximally. If we furthermore restrict the size of $S$ to the hardware word size $w$, we can efficiently implement the set operations on $visited$ by native bitwise operations, yielding a \emph{bit-parallel} implementation.

Note, however, that while this reduces the set operations to constant time, the loop iterating over all set bits of $visited[u]$ still needs $w$ iterations. So for sparse graphs there is a non-negligible overhead even though the rest of the algorithm graciously degrades to the single-source implementation in that situation. Even for adjacency matrix outputs, we cannot simplify the loop to a constant-time word move since choosing $S$ from topological levels means its nodes will in general not be contiguous in the output matrix. As a potential future extension, one could allow algorithms to permute the order of nodes in the output in order to implement this optimization.

\begin{algorithm}
  \begin{algorithmic}
    \Function{TLS}{S}
      \State find topological levels
      \For{$s \in S$}
        \State $next[level(s)] \gets next[level(s)] \cup \{s\}$
        \State $visited[s] \gets \{s\}$
      \EndFor
      \For{$l \gets \argmin_{s \in S} level(s), maxLevel$}
        \For{$u \in next[l]$}
          \For{$s \in visited[u]$}
            \State $reached[s, u] \gets true$
          \EndFor
          \State $visited[u] \gets \emptyset$
          \For{$v \in V, (u,v) \in E$}
            \If{$visited[v] = \emptyset$}
              \State $next[level(v)] \gets next[level(v)] \cup \{v\}$
            \EndIf
            \State $visited[v] \gets visited[v] \cup visited[u]$
          \EndFor
        \EndFor
        \State $next[l] \gets \emptyset$ \Comment Optimization: reset for next invocation
      \EndFor
    \EndFunction
  \end{algorithmic}
  \caption{Multi-Source TLS}
  \label{algo:MultiTLS}
\end{algorithm}
\section{Reverse TLS}

\section{Parallelized Implementations}

\section{Evaluation}

\section{Conclusion}

\end{document}
